% Living Systems --- LaTeX version

%latex 2.09:
\documentstyle[11pt]{article}

%This would be latex 2e:
%\documentclass[11pt]{article}
%\usepackage{a4}

\begin{document}
\title{Living Systems}
\author{Alexander Chislenko}
\date{}

\maketitle

\begin{abstract}

     The goal of this text is to show that there exist  common  system
laws  governing  the  development  of various complex objects, ranging
from animals and plants  to  sciences  and  economies,  regardless  of
details  of  their  implementation  in  different contexts, and try to
foresee future structural transformations of these objects.

     This is a manuscript that I am now  trying  to  discuss  with  my
friends,  while  thinking of what of it can go into print, and whether
it should.  I would greatly appreciate  any  response,  like  remarks,
references,  reading  and  contact advice etc., including all possible
notes on structure and language of the current text.  My dream  is  to
have some heavily commented copies of it returned.

     I would also gladly contact any person whose thoughts  wander  in
related areas.

{\centering              Alexander Chislenko \\
                         6 McLean pl. apt. 5  \\
                       Cambridge, MA 02140 USA  \\
                     Tel. (617) 864-33-82 (home)  \\}

\end{abstract}

\newpage

\tableofcontents

\newpage

\section{Introductory Note}

     In full accordance with written here, I believe that  no  article
can  represent  an  ultimate  value,  but is always only a part of the
growing organism of scientific knowledge that works partly through us,
writers  and  readers; and it is this entity, not us or articles, that
is slowly shaping itself towards the Absolute Truth.

     So we (I and this article) are trying to unselfishly take part in
this  process,  deliberately leaving some gaps in presentation and too
categorical, questionable or unexplained statements, with the  aim  to
provoke  further  discussion rather than to try prematurely polish our
own look.

     I have made no efforts to make this text an  easy  reading,  like
explaining  that  one  thousand  million  is a billion, or noting that
calling self-replication a necessary feature of all life  forms  would
be  insulting  for  lonely,  sick  and  elderly  people.  However, the
presentation is arranged so that the reader can get used to the ideas.


\section{Starting Point}

             The behavior of any object  depends  solely  on  its
        structure,  i.e.   the  number  of  its  elements and the
        character of their links.  This means that if we have two
        objects   with   equivalent   structures,  and  the  only
        difference between them is in implementation  and  naming
        of  their parts and links, then the behavior of these two
        objects will be exactly the same (except,  possibly,  the
        means of its expression).

             It is therefore reasonable to speak  of  classes  of
        objects  with  equivalent  structures;  let  us call them
        'systems'.


\section{Features Of System Structures}

     Human minds seem to be better  suited  for  spotting  differences
between  things  than  for  finding commonalities (it is not by chance
that some languages have  words  like  'pine'  and  'birch',  but  not
'tree').   This  may  be the main reason of why so little attention is
paid to some striking similarities between many natural and artificial
systems.


     This table briefly illustrates some common features and stages of
development of systems in the following three spheres:

\vspace{5mm}
\begin{tabular}{|l|l|l|l|} \hline
Feature/Sphere   &  Biology  &  Society & Computer Systems \\ \hline\hline
Starting point:  & unicellular  & primitive  & small single \\
simple and & organism & tribe & consecutive \\
unstructured (*) & & & machine \\ \hline
Mature system:  & multicell,    & interacting    & multitasking,  \\
interconnected  & multiorgan    & social         & networked      \\
substructures   & body          & institutions   & computers      \\ \hline
 Dual            & proteins      & people         & data and       \\
 basic parts     & and genes     & and memes      & instructions   \\ \hline
 Parasites       & microbes      & crime          &computer viruses\\ \hline
 Anti-parasites  & antibodies    & police         & anti-viral     \\
                 &               &                & programs       \\ \hline
 Rudimentary     & appendix,     & outdated laws  & obsolete       \\
 parts           & third eye     & \& institutions & data and code  \\ \hline
 Shared          & all tools as  & inter-state    & databases,     \\
 external        & shared limbs  & bodies and     & code libraries,\\
 units           & and sensors   & property       & netware,...    \\ \hline
 Self-awareness  & consciousness & Social-study   & system \& meta- \\
                 &               & institutions   & data \& software\\ \hline
 Intangible part & body/         & material life/ & hardware/      \\
 develops and    & consciousness & culture        & software       \\
 takes the lead  &               &                &                \\ \hline
\end{tabular}

\vspace{5mm}
 (*) in the sense of not consisting of parts of the same system level

     Some additional features are:  backup subsystems, spare parts and
capacities,  competition  of subsystems for resources and attention of
higher levels, conflicts over control, aging  problems  etc.,  etc\dots;
even  the [geometrical average] number of parts in all systems appears
to be the same.  These commonalities go far beyond basic  cybernetical
observations  of  information  processing; in fact, one may reasonably
suppose that there exist common system laws governing  development  of
morphic  entities regardless of the details of their implementation in
different contexts.

     All of the above listed features appear approximately at the same
levels  of  development  of each system; it seems that given the first
two columns of the above table and the [well-known] speed of  progress
in  computer  systems,  we would have been able as long ago as 1950 to
predict (and prepare for) the time of arrival of computer networks and
viruses   ---   instead   of   dreaming   of   big   lonely   MULTIVACs.
Totalitarianists  could  also  keep  in  mind   that   their   beloved
centrally-regulated   monolithic   societies,  just  like  unicellular
organisms and consecutive stand-alone computers, are competitive  only
as  long  as  they  (and their environments) are unsophisticated --- and
could think (or, rather, could have thought) of how to keep them  that
way.


     But it is of course easy to predict the past.





\section{Classes Of Systems}

     It seems possible to divide systems into  general  classes,  like
colony,  organism and biocenosis (ecology), judging by their essential
features,  most  important   of   which   seem   to   be   complexity,
interconnectedness,  clusterization, strength of feedbacks, etc.  If a
certain system structurally falls into a given class, say, 'organism',
then  {\bf it  is an organism}, since God/Nature perceive(s) and guide(s) it
as such, while we can call it a rabbit, piece  of  software,  national
economy,  language,  culture  or  science.   The  difference between a
structurally equivalent computer program and a body (or a culture  and
an  intelligent  consciousness) is no more than in our perception.  We
think we 'see' 'real' living organisms and, hence, they 'exist', while
other  systems  are just 'imaginary entities' (though we never see the
structures, but only external properties,  like  skin  color,  spatial
position  etc.,  and  all  important features have to mentally derive,
anyway).

     We can look around and see that our planet is abundant  with  all
kinds of developing life and intelligence, and for much of it we serve
as  (still,  unfortunately,  unconscious)   carriers   or   elementary
substrata.

     All our subsystems and supersystems live their own lives and have
little  (though  gradually  growing)  interest  and  knowledge of each
other.

     We have little understanding of how our own cells or the parts of
consciousness  work,  and  no single individual is likely to ever have
much more, since the complexity of these systems by  far  exceeds  the
capacities of the consciously controlled part of the human brain.  But
there are  other  entities,  namely,  the  sciences  of  citology  and
psychology, that, once brought to life, continue working as our agents
of communication with our own parts,  studying  and  testing  them  in
their  own  planned  and orderly manner, and often pursuing goals that
represent their own logic of development rather  than  the  individual
understanding of practical value by any single human being.



\section{Our Perception Of Other Living Systems}

\subsection{Familiar Entities}

     We accept other systems as living if  they  look  just  like  us,
especially  if  they  have the same appearance, age, language and skin
color.  (This is why much more attention is paid to unborn or mentally
retarded  representatives  of Homo Sapiens than to much more conscious
healthy adults of other species).



\subsection{Alien Entities}

     We have special names (just like 'birch' and  'pine')  for  those
systems [we think] we understand and use special methods for each type
without trying to generalize  or  recognizing  equal  rights  for  all
living entities.  Usually we take it for granted that cells, cultures,
subconscious parts of our minds, etc., work, but do not  care  whether
and  how  they feel.  The Green movement is now broadening its ethical
view, starting to think of the rights of insects, bacteria and  baking
yeast.   But  there are some things they do ignore:  namely, that such
entities as philosophy or the Green movement  itself  are  more  alive
than  the  baking  yeast,  in  any  reasonable sense of the word; that
bed-bugs, cock-roaches, mafia  and  urban  life-style  are  also  very
natural formations, and still the first three of them (or all four, by
some opinions) are well worth destroying; that  life  is  a  permanent
interaction   among   all   living   beings,  and  some  of  them  do,
unfortunately, have to  lose  (I  am  far  from  saying  they  may  be
thoughtlessly destroyed).

     It seems that the process we see unfolding around  us  is  not  a
destruction  of nature by 'non-nature', but a gigantic leap in natural
evolution, with an explosive number of new  cultural,  scientific  and
technological  species  pushing  some old species (like, alas, spotted
owl, and, hopefully, bed-bugs) out of the  niches  of  their  original
existence,  and we have the greatest historical opportunity to observe
this process, take part in guiding it --- and reap  incredible  benefits
for ourselves.



\subsection{Complex Systems}

     If a thing is complex enough, and its analysis  seems  to  be  of
little practical value, people prefer to perceive/admire/worship it as
a 'whole', often hating the very idea of  'disassembling'  the  sacred
object.   Music  and arts are good examples of this situation.  Though
there is no such thing as an undifferentiated whole (with, possibly, a
few  elementary  exceptions),  and  those  nice  objects would be much
better off if there were a place in their  admirers'  minds  for  both
their beauty and design.



\subsection{Supersystems}

     If a  system  is  incredibly  complex  and  its  ways  cannot  be
understood  to any reasonable degree, its influence can be sensed only
by certain events that tend to  occur  when  we  follow  some  vaguely
understood  rules.   In this case, it is easier to simply 'obey' those
rules, i.e.  to do things expected to  bring  positive  feedbacks,  or
just  try  to  pray  to the supersystem in one's own language and wait
until it hears and helps.  This is certainly very similar to  what  we
think  of  God.  Let us also look at it from another point of view and
imagine what our cells might think about the systems  they  are  parts
of.   All  of  them could agree that a system consisting of themselves
(and hence not a material object, but, like a society for us,  just  a
structure  they  support, a way they do things together) cannot think,
feel or plan anything by itself.  "It may look structurally like it is
alive,  but  it is us who do it all", they might say of us --- as people
say of grander systems.  Most of the cells are unable to  collect  any
information   about  their  environment  beyond  what  is  needed  for
performing their simple physical duties.  They keep doing their  jobs,
exhausting available resources, assuming that before the resources are
over, some smarter neighbors or higher powers will take  care  of  the
problem.   Those with higher abilities (but still in no way capable of
comprehending the Great Whole --- e.g., a hair root they live  in)  take
part  in  'social'  development and may choose between improving their
environment themselves and sending complaints to some higher  'divine'
authority.   The  results  of  the  first  approach are guaranteed but
limited by the cell's model of  the  environment  and  abilities;  the
second approach is not always successful, since the gods are not quite
almighty and tend to leave the complaints misunderstood or unnoticed ---
but  if  the complaints are loud enough and/or supported by neighbors,
they may be gracefully acknowledged.  In our example, some  substances
are  sent  in by local gods(=organs), or the person whose head carries
the hair, scratches or washes it, or ---  an  extremely  rare  case!   ---
there  can  be  a  direct interference from some super-authority, like
killing of this particular rioting cell by a well-focused X-ray  beam,
to  stop  the trouble.  It is hardly possible to give a general advice
of what can be asked by a part from a whole, but  it  is  possible  to
note  how.   By  utilizing  the  part's  links with the system; silent
prayers are never heard, though they  may  unintentionally  alter  the
internal  state of the subsystem and bring desirable --- or some other ---
results.  They may also employ some hidden information channels,  like
'talking'  to  other layers of one's own consciousness or transmitting
weak radio messages.



\section{Evolution Of A Supersystem}

     At the very beginning, parts voluntary form some joint structure,
following  their own logic of development.  This is already a birth of
a supersystem, though it is still weak and vitally  dependent  on  its
creators;  its  life seems to be in the hands of a few of them.  Then,
after some time, it grows to become their more or less equal  partner,
with  its  own  needs  and  logic,  but  still  understandable  by the
conventional wisdom of the parts --- if they choose to study and respect
the  laws  of  the  new  formation  they found themselves in.  At this
stage, they may choose to occupy one or another niche in  the  system.
They  are  still  more  or  less  free,  counting personal balances of
participation in different kinds of activities ---  a  market  phase  of
development.   At  further  stages,  with  new  strengths  and runaway
complexity, the new superorganism develops its own ideas of  progress.
Its  creators  lose track of its development, but go on taking part in
it, since now they are supported by the system for their loyalty --- and
punished  for lack of it.  An integrated system can seldom abandon its
parts, even if they are inefficient;  rather  it  provides  them  with
additional  resources and attention.  So the competition for resources
and records of local accounts lose their leading roles to more complex
and  distributed  decision-making schemes.  Meanwhile, parts see their
safety and wealth rapidly increase at the expense  of  their  freedom,
until  none  of  them  is  able to abandon the system and live 'in the
wild'.  This is what happened to our cells.  This tends to  happen  in
totalitarian societies, with their overdeveloped social structures (or
just well developed, from their own point of view), in well-integrated
corporations, in software packages, in happy families and is gradually
happening in advanced economies.   "The  Futurist"  once  discussed  a
possibility  that technology can turn humans into cyborgs --- creatures,
'consisting by more than a half of artificial parts'.  I would  define
a  cyborg  nation  as  one  with  more  than  half  of its individuals
artificially kept alive.  Technology has really made us healthier  and
better  off  ---  within  the  economic system, but how many of us would
survive if all of it suddenly disappeared?  Hardly half would be left.
And  not  only because we are unprepared; there are simply too many of
us, and it is likely that more than 95\% of us would rapidly die out in
"natural"  (?!)  conditions  --- and this percentage has been increasing
throughout all of the human history --- with an obvious upper limit.


\section{Our Attitude To Supersystems}

     The prefix 'super' in this context means no more than a shorthand
for  'upper-level'  and  reflects  our view on the system from inside,
rather than from outside.  So, 'super' means just a metaphenomenon and
shouldn't  be  confused  with  'superior'.   In  fact, the upper-level
structure of a system can be much simpler than the internal structures
of  many  of  its  parts.   Hence,  we  should not be shocked that the
systems we are in --- like a scientific  community,  United  Nations  or
Gaia  --- are, at the first stages of their development, inefficient and
often waste the precious resources they are going  to  badly  lack  in
their  near  future.   We  may  learn  what  it feels to live within a
primitive  beast  ---  or,  rather,  a  child.    Children   are   often
inexperienced,  clumsy  and  short-sighted --- but they grow, especially
when properly taught.



\section{Networks And Ecologies}

     The now popular word 'network' doesn't describe a special type of
systems,  but  just  means  a  multiplicity  of relatively stable ties
between neighbor elements, i.e.  the 'local complexity' of the system,
and  hints  at presence of multiple feedbacks --- a feature also present
in many other types of structures.

     There were free-trade networks of merchandise flow between tribes
and  city-states  in  ancient  times;  later  they  were  replaced  by
non-network,  but  much  more  complex  and   efficient   systems   of
international  trade,  though  we  can  still  find  their remnants in
grey-market structures.

     There also are highly deterministic  governmental,  computer,  or
sewage  networks  that  demonstrate  predictable  responses  for known
inputs  and  are  different  from  plain  tree-like   hierarchies   in
complexity, but not in nature.

     And there are flexible ecological networks based  on  cooperation
and   commitment   (though   quite   not  free  from  conflicts),  and
instinctive,  if  not  conscious,  feeling  of  common  destiny.   But
networking is not the most important word here; a scientific community
corresponding through one central computer is much easier to  call  an
ecology  than, say, a complex early-warning defence network.  The real
difference here lies in the nature of the links rather than  in  their
number.

     Among different system  types,  biocenotic  (ecological)  systems
seem superior in many respects:  they combine greater freedom of parts
with overall complexity, flexibility and stable perfomance within very
wide limits of tolerance.  But all planning in an ecology is local and
is performed by non-ecological participating entities.  And the  human
mind  seems  to  resist the very idea of creating a [social] system it
cannot directly control...



\section{Consequences Of The Living Systems View}

     Suppose we can really look at many things as living entities; but
what difference does it make?


     First, it attemts to present a  common  basis  for  holistic  and
rational  paradigms; it might be called 'demystified holism'.  Second,
it seems to be a platform for building a kind  of  transhuman  ethics.
Third,  we  can  try  to formulate some generic methods for contacting
other living --- and thinking --- entities.



\section{A Scenario For Further Social Development}

     Most existing objects employ a combination of  various  structure
types   for  performing  different  functions.   Therefore,  it  seems
reasonable to build the classification of structures on  the  analysis
of system functions, rather than objects themselves.

     Let us consider three basic types  of  system  functions  on  the
examples of social and biological structures:

\vspace{5mm}
\begin{tabular}{|l|l|l|l|} \hline
 Feature \ Function & Life support &  Regulation &    Development      \\ \hline\hline
 Social structures & Market system& Institutions& Community           \\  \hline
 Biological systems& Free colony  & Organisms   & Symbiosis to ecology\\  \hline
 Control methods   & Facilitation & Direct      & Participation       \\
                   &              & interference&                     \\  \hline
 Oscillation       & Smooth,      & Cubistic,   & Slight intricate    \\
   patterns        &  wave-like   & very regular& oscillations (within\\
                   &              & cycles      & limits of tolerance)\\  \hline
 Phase portraits   & Periodic     & Quasy-linear& Chaotic attractors \\
                   &  attractors  & shapes      &                     \\  \hline
\end{tabular}

\vspace{5mm}
     Let us try to assess future social roles of these three types  of
functions:

\begin{itemize}
\item   traditional life-support, such as local trade and services, is
likely  to be provided by good old market forces well into foreseeable
future, and institutional and technological influence is not likely to
translate into any drastic structural change in this sphere.

\item     regulatory functions will be carried  out  by  the  developing
network  of  institutional hierarchies --- with some structural changes,
especially on the global level; here we should beware  of  cancer-like
growth  of simple centralized structures, despite of anti-totalitarian
shots timely made by History into some parts of the developing  global
social  organism (I mean painful socialist experiments in the East) to
make the rest of it more  resistant  to  evilly  attractive  ideas  of
centralization.

\item    structural social and technological changes --- the  essence  of
the  coming  age  ---  will be provided by more and more biocenosis-like
systems;  hence,  we  can  expect  corresponding  changes   in   phase
portraits,  such  as  economic  cycles.   The  innovations  have never
appeared in the society by means of free exchange  or  direct  orders,
that  can  only  take  care  of  allocation of necessary resources and
spreading the novelty.  With innovations  becoming  the  core  of  the
social   life   we   will   see  increasing  role  of  non-market  and
non-governmental groups with common interests and knowledge in various
special  areas.   We  can  also expect that economic (= monocriterial)
considerations will  continue  losing  their  indicative  and  guiding
roles.
\end{itemize}

     In the evolution of  any  system,  new  parts  appear  to  better
reflect  existing  (and  planned)  functions;  ideally,  the  system's
structure should reflect its functions; we can observe that  functions
are  getting incorporated into new structural entities:  separation of
flowers from leaves;  arts,  advertisement  and  money  from  material
production;  information  from  its  carriers,  etc.   This transition
becomes  possible  after  new  features  become  ripe  enough  to   be
acknowledged,  or  the old ones grow complex enough to require further
structurization, or there appear breakthroughs in the speed  and  ease
of internal communications system.

     In the socio-economic  sphere  it  means  that  while  the  walls
between  functionally homogeneous entities, until recently represented
by high costs of material and  information  interchange,  continue  ot
crumble,  thus  altering the effective metrics of the social space, we
will see more and more institutions based on similarity  of  interests
of  their  members  rather  then  on their spacial proximity, with the
old-fashioned organizations based  on  territorial  sovereinty  rights
(state and local governments) rapidly losing their roles.  Among other
things, it means that the readers with investments  into  local  media
might win from moving them into special interest editions.



\section{Seeds For Thought}

\begin{itemize}
    \item How do we teach other systems:  by training and  testing  them;
just  as  we  do with children, with our own cells and organs (morning
exercises),  with  parts  of  our  minds  (lessons  and  tests),  with
computers,  sports  teams,  military divisions, etc.  --- by giving them
relatively small but structurally important tasks.  All these  spheres
can  share  their  experience.   ---  Can  we  teach  UN using cognitive
psychology?

    \item An analogy  between  social/scientific  development  and  yoga:
trying to bring consciousness into every cell of the body

    \item The ultimate goal of evolution:  God --- through  integration  of
all  systems  from  quantum to cosmic scale and harmonization of their
relations?  And how do we know when it is (or whether is  has  already
been) achieved?
\end{itemize}


\section{Final Note}

     Let us try to guess what other complex  entities  could  do  when
they  at  last  develop  some  real intelligence and attempt to openly
contact us, humans.  A good idea would be  to  start  with  a  message
carrying  a  manifest of their conscious life.  The only way of making
such a message understandable to us would be to pass  it  through  our
regular  information  channels,  like  mind  patterns  or this printed
article.  And, as always, all of the  agencies  employed  in  carrying
this  message  ---  philosophy, language, society, the human author, his
neurons, word processor, U.S.  Postal Service and others --- perceive it
as  belonging to them.  The question is how many of them are developed
enough to do it consciously?  While writing this text, I had  a  queer
sensation  of  someone  guiding my hand.  Could it be one of them?  Or
someone else?..


\section{Habitats For Life}

     Each of us has read something about  possibilities  of  non-human
intelligence  and  'even'  non-carbon  life;  but I failed to find any
interesting ideas about general conditions necessary to support  life.
I  would  close  the  problem  of intelligence by saying that it is an
attribute common to all life-forms, with differences in  features  and
sophistication.  We can also more or less easily define such things as
a reflective consciousness (as a feeling of a  system  of  itself,  or
storage  of  the  information  on  the global state of the system on a
local level, etc.) and other attributes  of  intelligence  in  general
system terms.

     I wouldn't like to argue here about the definition of life; there
is  a  number  of  them,  and  even a few general features that almost
everyone seems to agree on, sound very questionable.  Thus, growth and
self-replication  apply  to chrystals, rumors, puddles and clouds, but
not to elderly people\dots  Instead, I would suggest a view that  'life'
is  a  certain  stage  of  developmental  processes, and all processes
inevitably come to  (and  pass)  this  stage  provided  certain  basic
conditions  are  met.   So  let's try to discuss these conditions, see
what kind of environments can provide them, and what  changes  in  the
conditions  can  lead  to the development of really (i.e.  in essence,
not implementation) different forms of life and intelligence.

     Definition  of  terms.   In  mathematics,  a  space   ---   whether
topological,  linear  or  other  ---  is  just  a  set  of points with a
structure; here  let's  think  of  a  'space'  as  the  basic  uniform
structure   of   the  environment,  and  'objects'  or  'entities'  as
[epi]structures 'immersed' in  this  space.   The  readers  with  some
understanding  of  mathematics can derive stricter meaning themselves,
and those without it won't understand it anyway\dots

     Here are the basic 'life conditions' --- i.e.   attributes  of  the
environment that ensure that the development will go far enough:

\begin{enumerate}
     \item Large enough  living  space  ---  in  terms  of  the  number  of
elementary units.

     \item Ability of units/elements to interact

     \item Relative stability of structures (links between elements) --- so
that new emerging entities could survive long enough to communicate to
others and/or give life to new forms.

     \item Relative mobility of structures --- to ensure that  some  change
can occur.

     \item Long enough life time in terms of relative units of time
\end{enumerate}

     I do not claim that this is  a  complete  set  of  necessary  and
sufficient  conditions  for  life  development  (which doesn't mean it
isn't), and they are not even  strictly  defined  here,  but  this  is
already  enough  to  present  one  more system table, and draw certain
conclusions.

\vspace{5mm}
\begin{tabular}{|l|l|l|l|l|l|} \hline
 Space     &   elements  &  size   &stability& mobility  &  time  \\ \hline\hline
 planetary & Atoms       & 10*40?  & Good    & fair      & excel. \\
 surface   & (C,O,H,etc.)&         &         &           &        \\ \hline
 planetary & Atoms       & 10*60?  & Fair    & fair      & excel. \\
 body      & (Si,O,etc.) &         &         &           &        \\ \hline
 Universe  & Galaxies    & 10**11  & Poor    & Good      &  Bad   \\ \hline
 Universe  & Vacuum      & Perfect &  ?      &   ?       & Perfect\\
           &Fluctuations & 10**200 &         &           &        \\ \hline
 science   & bits / memes& Fair    &  Good   & Good      & Good   \\
           &     /ideas  & 10*20 ? &         &           &        \\ \hline
 Star      &  plasma     & Good    &  Fair   & Poor      & Good   \\
           &  currents   & 10**40? &         &           &        \\ \hline
 Sea       & Waves       & Fair    &  No     & ---       & Good   \\
           &             & 10**20  &         &           &        \\ \hline
\end{tabular}

\vspace{5mm}
     The  space  structure  and  interaction  details  have  a  strong
influence  on  all  structures  living in the given environment; let's
just sketch some differences that can arise in other environments:

     If the space allows teleportation, we can expect to see  bodies
consisting  of  physically  unconnected  parts, more ethical societies
(since otherwise they would be simply killed by  the  omnipresent  and
uncathable crime), impossibility of totalitarianism (which is strongly
dependent on the ability of the government to keep the  citizens  from
getting  out  of  its  control  area), absence of cities and other big
clusters of population (all places are equally close to  each  other),
etc.

     We can also note that all our bodies, social  organizations,  and
even concept structures reflect the structure of the space we live in;
research shows that the average number of parts  in  a  whole  in  all
human  classifications  is  very  close  to  number pi, which seems to
reflect the relations of proximity of different  parts  of  the  human
brain, as well as individual neurons, living in the Cartesian space.

     Here it is probably time to note that spaces in which many  other
structures  reside,  are  effectively non-Cartesian.  For example, the
computer memory space is essentially a discrete set of points, and  it
takes [approximately] the same amount of time to get from one [memory]
location to another.

     The success of transportation and communication systems is  based
on  their  changing  the effective metrics of social space; originally
designed to shorten the distances  between  the  [participants],  they
often  entirely change the structure of the space they operate in; for
example, effective structures  of  telecommunications  networks  (with
time  and  ease  of  access  as  measures of distance) are essentially
discrete,  and  not  only  our  regular  notions   of   distance   and
dimensionality  are  irrelevant here, but the whole telecommunications
service industry operates according to new laws:  there are  no  local
monopolies,  at  least  within a calling area, there is no need to put
the communications equipment on  the  cross-roads  (or  even  keep  it
visible   in  the  physical  world);  the  market  niches  are  purely
functional, without territorial considerations; there is no such thing
here  as  differential  rent;  the  number  of  competing  services is
different from one in regular 3-D businesses, which affects the degree
of monopolization in the industry, decision-making processes, business
cycles,  etc.   One  of  the  major   reasons   of   transition   from
decision-making  based  on  participatory  democracy to representation
schemes and multilevel bureuacracies is the limitations  of  Cartesian
space  that  do  not allow sufficiently large assemblies of citizens -
namely, inability of two objects to occupy a single space, and  travel
expenses  ---  and  these  limitations simply do not exist in many other
spatial structures.

     Most of our concept structure is based  on  our  3-D  notions  of
localities,  weights, shapes and appearances, and there seem to be few
people and concepts that ever leave the limitations of  the  Cartesian
space,  or  generalize enough to make any sense outside our particular
spacial implementation of functional structures.


[ to be finished ]
 
\end{document}
